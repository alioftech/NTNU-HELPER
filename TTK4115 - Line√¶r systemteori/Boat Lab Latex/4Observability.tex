\newpage
\section{Observability}

%%%%%%%%%%%%%%%%%%%%%%%%
% 4.1 Derivation of State Space Matrices
%%%%%%%%%%%%%%%%%%%%%%%%
\subsection{Derivation of State Space Matrices}
The derivation of the State Space Matrices is done with the state vector, input and disturbance as shown here:
\begin{equation}\label{state vector}
\begin{array}{ccc}
     x = \left[ {\begin{array}{*{20}{c}}
{{\xi _\omega }}\\
{{\psi _\omega }}\\
\psi \\
r\\
b
\end{array}} \right],&  
    u = \delta  ,&  
    \omega  = \left[ {\begin{array}{*{20}{c}}
    {{\omega _\omega }}\\
    {{\omega _b}}
\end{array} }\right] 
\end{array}
\end{equation}

We have the following model:
\begin{align}
    \begin{array}{l}
{{\dot \xi }_2} = {\psi _\omega }\\
{{\dot \psi }_\omega } =  - {\omega _0}^2{\xi _\omega } - 2\lambda {\omega _0}{\psi _\omega } + {K_\omega }{\omega _\omega }\\
\dot \psi  = r\\
\dot r =  - \frac{1}{T} + \frac{K}{T}(\delta  - b)\\
\dot b = {\omega _b}\\
y = \psi  + {\psi _\omega } + v
\end{array}
\end{align}

The system can be written as:

\begin{equation}
\begin{array}{cc}
    \dot{x} = \left[ {\begin{array}{*{20}{c}}
{{x_2}}\\
{ - {\omega _0}^2{x_1} - 2\lambda {\omega _0}{x_2} + {K_\omega }{\omega _1}}\\
{{x_4}}\\
{ - \frac{1}{T}{x_4} - \frac{K}{T}{x_5} + \frac{K}{T}u}\\
{{\omega _2}}
\end{array}} \right] ,&  
    y = {x_2}{x_3} + v  
\end{array}
\end{equation}

This corresponds to a system on the following form:

\begin{equation}
   \begin{array}{l}
\dot x = Ax + Bu + E\omega \\
y = Cx + v
\end{array}
\end{equation}

with matrices:
\begin{equation*}\label{matrices}
   \begin{array}{cccc}
       A = \left[ {\begin{array}{*{20}{c}}
0&1&0&0&0\\
{ - {\omega _0}^2}&{ - 2\lambda {\omega _0}}&0&0&0\\
0&0&0&1&0\\
0&0&0&{ - \frac{1}{T}}&{ - \frac{K}{T}}\\
0&0&0&0&0
\end{array}} \right] ,&
       B = \left[ {\begin{array}{*{20}{c}}
0\\
0\\
0\\
{\frac{K}{T}}\\
0
\end{array}} \right] ,& \\\\
       E = \left[ {\begin{array}{*{20}{c}}
0&0\\
{{K_\omega }}&0\\
0&0\\
0&0\\
0&1
\end{array}} \right] ,&
       C = \left[ {\begin{array}{*{20}{c}}
0&1&1&0&0
\end{array}} \right]
   \end{array}
\end{equation*}

To check the observability of the system, we use these following parameters that were calculated in previous sections:

\begin{equation}
   \begin{array}{l}
K = 0.1742\\
T = 86.5268\\
{\omega _0} = 0.7823\\
\lambda  = 0.0862
\end{array}
\end{equation}


%%%%%%%%%%%%%%%%%%%%%%%%
% 4.2 Observability without Disturbances
%%%%%%%%%%%%%%%%%%%%%%%%
\subsection{Observability without disturbances}
Without disturbances, both the state vector from equation  (\ref{state vector}) and the matrices A and C are rewritten as:

\begin{equation}
   x = \left[ {\begin{array}{*{20}{c}}
\psi \\
r
\end{array}} \right]
\end{equation}

\begin{equation*}
   \begin{array}{cc}
       A = \left[ {\begin{array}{*{20}{c}}
0&1\\
0&{ - \frac{1}{T}}
\end{array}} \right] ,&  
       C = \left[ {\begin{array}{*{20}{c}}
1&0
\end{array}} \right]  
   \end{array}
\end{equation*}

This provides an observability matrix which has a rank of 2, meaning the system is observable.

\begin{equation}
   \mathbf{\math\mathcal{O}} = \left[ {\begin{array}{*{20}{c}}
1&0\\
0&1
\end{array}} \right]
\end{equation}


%%%%%%%%%%%%%%%%%%%%%%%%
% 4.3 Observability with current disturbance
%%%%%%%%%%%%%%%%%%%%%%%%
\subsection{Observability with current disturbance}
When current disturbance is applied, it is necessary to rewrite both the state vector and the matrices A and C again, on the form:

\begin{equation}
    x = \left[ {\begin{array}{*{20}{c}}
\psi \\
r\\
b
\end{array}} \right]
\end{equation}

\begin{equation}
    \begin{array}{cc}
       A = \left[ {\begin{array}{*{20}{c}}
0&1&0\\
0&{ - \frac{1}{T}}&{ - \frac{K}{T}}\\
0&0&0
\end{array}} \right] ,&  
       C = \left[ {\begin{array}{*{20}{c}}
1&0&0
\end{array}} \right] 
    \end{array}
\end{equation}

The observability matrix for the system with current disturbance is given by (\ref{observer}), and by using the MatLab command $rank(Ob)$, we can see that it has full row rank of 3, meaning the system is observable.

\begin{equation*}
    \mathbf{\math\mathcal{O}} = \left[ {\begin{array}{*{20}{c}}
1&0&0\\
0&1&0\\
0&{ - \frac{1}{T}}&{ - \frac{K}{T}}
\end{array}} \right]
\end{equation*}

\begin{equation}\label{observer}
    \mathbf{\math\mathcal{O}} = \left[ {\begin{array}{*{20}{c}}
1&0&0\\
0&1&0\\
0&{ - 0.0116}&{ - 0.0020}
\end{array}} \right]
\end{equation}


%%%%%%%%%%%%%%%%%%%%%%%%
% 4.4 Observability with wave disturbance
%%%%%%%%%%%%%%%%%%%%%%%%
\subsection{Observability with wave disturbance}
With wave disturbance is applied, the state vector and the matrices A and C are rewritten on the form:

\begin{equation}
    x = \left[ {\begin{array}{*{20}{c}}
{{\xi _\omega }}\\
{{\psi _\omega }}\\
\psi \\
r
\end{array}} \right]
\end{equation}

\begin{equation}
    \begin{array}{cc}
        A = \left[ {\begin{array}{*{20}{c}}
0&1&0&0\\
{ - {\omega _o}^2}&{ - 2\lambda {\omega _0}}&0&0\\
0&0&0&1\\
0&0&0&{ - \frac{1}{T}}
\end{array}} \right] ,&  
       C = \left[ {\begin{array}{*{20}{c}}
0&1&1&0
\end{array}} \right]
    \end{array}
\end{equation}

Using the same MatLab command as in the previous section, we find that the observability matrix has a rank of 4, meaning that this fourth order system is observable.

\begin{equation*}
    \mathbf{\math\mathcal{O}} = \left[ {\begin{array}{*{20}{c}}
0&1&1&0\\
{ - {\omega _o}^2}&{ - 2\lambda {\omega _0}}&0&1\\
{2\lambda {\omega _0}^3}&{4{\lambda ^2}{\omega _0}^2 - {\omega _0}^2}&0&{ - \frac{1}{T}}\\
{ - {\omega _0}^4(4{\lambda ^2} - 1)}&{ - 8{\lambda ^3}{\omega _0}^3 + 4\lambda {\omega _0}^3}&0&{\frac{1}{{{T^2}}}}
\end{array}} \right]
\end{equation*}

\begin{equation}
   \mathbf{\math\mathcal{O}} = \left[ {\begin{array}{*{20}{c}}
0&1&1&0\\
{ - 0.6120}&{ - 0.1239}&0&1\\
{0.0825}&{ - 0.1349}&0&{ - 0.0116}\\
{0.3635}&{0.1626}&0&{0.001}
\end{array}} \right]
\end{equation}


%%%%%%%%%%%%%%%%%%%%%%%%
% 4.5 Observability with wave and current disturbances
%%%%%%%%%%%%%%%%%%%%%%%%
\subsection{Observability with wave and current disturbances}

With both wave and current disturbances, we use the state vector from (\ref{state vector}) with corresponding matrices from section 4.1 to obtain the observability matrix shown in (\ref{ob5}). This observability matrix has a rank of 5, meaning this fifth order system is observable.

\begin{equation*}
    \mathbf{\math\mathcal{O}} = \left[ {\begin{array}{*{20}{c}}
0&1&1&0&0\\
{ - {\omega _0}^2}&{ - 2\lambda {\omega _0}}&0&1&0\\
{2\lambda {\omega _0}^3}&{4{\lambda ^2}{\omega _0}^2 - {\omega _0}^2}&0&{ - \frac{1}{T}}&{ - \frac{K}{T}}\\
{ - {\omega _0}^4(4{\lambda ^2} - 1)}&{ - 8{\lambda ^3}{\omega _0}^3 + 4\lambda {\omega _0}^3}&0&{\frac{1}{{{T^2}}}}&{\frac{K}{{{T^2}}}}\\
{8{\lambda ^3}{\omega _0}^5 + 4\lambda {\omega _0}^5}&{16{\lambda ^4}{\omega _0}^4 + 12{\lambda ^2}{\omega _0}^4 + {\omega _0}^4}&0&{ - \frac{1}{{{T^3}}}}&{ - \frac{K}{{{T^3}}}}
\end{array}} \right]
\end{equation*}

\begin{equation}\label{ob5}
   \mathbf{\math\mathcal{O}} = \left[ {\begin{array}{*{20}{c}}
0&1&1&0&0\\
{ - 0.6120}&{ - 0.1349}&0&1&0\\
{0.0825}&{ - 0.5939}&0&{ - 0.0116}&{ - 0.0020}\\
{0.03635}&{0.1626}&0&{0.0001}&{2.3 \cdot {{10}^{ - 5}}}\\
{ - 0.0995}&{0.3415}&0&{ - 1.5 \cdot {{10}^{ - 6}}}&{ - 2.7 \cdot {{10}^{ - 7}}}
\end{array}} \right]
\end{equation}
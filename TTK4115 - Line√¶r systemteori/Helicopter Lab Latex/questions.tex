\section{Why linerize?}
Linearization around an equilibrium point (where the derivative of the full state vector is zero) tells you how the system behaves for small deviations around the point.

Keep in mind, however, that a linearized model is valid
only when the system operates in a sufficiently small range around an equilibrium point.

The linearization , also referred to as a small-signal model, is valid only in asufficiently small neighborhood of the equilibrium point $x^∗$

Linearization is needed to design a control system using classical design techniques, such as Bode plot and root locus design. Linearization also lets you analyze system behavior, such as system stability, disturbance rejection, and reference tracking.




\section{Why use transfer functions}
Represent linear time-invariant systems in the frequency domain

A transfer function is a convenient way to represent a linear, time-invariant system in terms of its input-output relationship. It is obtained by applying a Laplace transform to the differential equations describing system dynamics, assuming zero initial conditions. In the absence of these equations, a transfer function can also be estimated from measured input-output data.


\section{Case 3 - Reaksjon på arbeidsfordeling}

\subsection*{Situasjon}
Det er den 7. landsbydagen og alt ligger til rette for at dagen i dag skal brukes til konstruktivt arbeid. I ukene i forveien har det blitt lagt vekt på å bli kjent med hverandre, kartlegge våre ambisjoner for prosjektet samt bli enige om konsept vi skal jobbe med i ukene framover. Disse ukene har vært greie, men nå er tålmodigheten vår tøyd et stykke og alle ser fram til å begynne ordentlig på prosjektet. Dessverre er ikke gruppa fulltallig, da Ali er fraværende på grunn av sykdom.\\
\\Da vi bestemte oss for konseptet vi skulle arbeide med, ble det foreslått av veileder at vi i tillegg til en prototype også skulle se på hva slags samfunnsrelaterte oppgaver og utfordringer som måtte løses for at Truck Platooning skulle bli en realitet i framtiden. Dette syntes vi hørtes fornuftig ut og gruppa ble derfor delt i to subgrupper. Tre medlemmer av gruppa skulle konsentrere seg om prototype-utvikling, mens de to andre skulle konsentrere seg om de samfunnsrelaterte utfordringene. 
\\Ettersom vi var en ganske todelt gruppe med tanke på bakgrunnene våre, falt arbeidsfordelingen seg ganske naturlig. Håkon og Lars skulle ha hovedansvaret for prototyping, da denne arbeidsoppgaven krever innsikt og erfaring i programmering og datateknologi. Dette passet de veldig bra, ettersom de studerer kybernetikk og elektronikk. Ali skulle bistå  begge sub-gruppene da han også er fra kybernetikk, men hadde interesse for begge temaene. Den andre subgruppa består da av Jonas og Anders som begge kommer fra maskin-linja.\\

\\Det kan spores noe missnøye eller skuffelse hos noen av medlemmene angående denne arbeidsinndelingen. Jonas valgte denne landsbyen siden han er interessert i data og IT og ønsket å jobbe innen dette fagfeltet i dette prosjektet. Jonas er derfor litt skuffa når han innser at det blir mer samfunnsrelatert arbeid som skal gjøres, og ikke så mye av det tekniske han kommer til å jobbe med i dette prosjektet. Da denne arbeidsfordelingen blir bestemt, ca en uke før denne landsbydagen, sier Jonas ifra om dette. Men, han svelger kamelen da han er enig i at det bør være en samfunnsdel av prosjektet, og den mest naturlige inndelingen er at Håkon, Ali og Lars jobber med prototypen. Han ytrer at han har erfaring fra å lage prototyper tidligere i utdanningen, men innser at mangelen på IT- og elektronikkforståelse begrenser ham her.\\
\\
Ettersom vi er to grupper som holder på med vidt forskjellige deler av samme prosjekt ble det foreslått av Håkon den foregående uken at vi bruker programmet Trello, se appendiks B. Dette er et organiseringsverktøy som muliggjør bruk av SCRUM-modellen, slik at vi alle har oversikten over hva hver enkelt jobber med. Tanken bak dette er å gjøre det oversiktlig for alle å se hva hverandre holder på med. Uken før  gikk dermed til å bestemme spesifikke arbeidsoppgaver og fylle de inn i Trello, slik at vi denne uka kunne begynne å jobbe med de.\\

\\Etter dagens innsjekk påbegynnes arbeidet med oppgavene vi bestemte oss for uken før. Det senker seg en ro blant gruppa da alle er fokusert på å vise at vi kan holde ord og løse de oppgavene vi er satt til å gjøre. Det blir stille da arbeidet starter. Jonas sitter med musikk på øret og leser seg opp på statistisk sikkerhet, Anders er dypt konsentrert av å forstå hvilke parametere som bidrar til luftmotstand, Ali gjør research på European Truck Platoon Challange, Håkon leser seg opp på Nordic Semiconducter sin chip, mens Lars ser etter løsninger for hvordan vi skal svinge og styre prototypen. 
Rett før lunsj kommer fasilitator Martine med en bemerkning; \\

\textit{"Dere må slutte å være så stille. Det mye morsommere å fasilitere når dere ikke jobber så hardt."}{--Martine}\\ 

Etter lunsj fortsetter vi der vi slapp og går til hvert vårt arbeid. Anders bemerker seg at det er veldig lite kommunikasjon innad i gruppa, men tenker at det er naturlig ettersom vi holder på med vidt forskjellige ting. Jonas ser opp fra skjermen og kommer med en snodig bemerkning om hvor langt en flåte med autonome biler må kjøre før man med sikkerhet kan si at det er like sikkert som en vanlig sjåfør. Gjennom dagen kommer det noen få slike bemerkninger, men ellers er det veldig stille da alle er tilfreds med arbeidet vi gjør hver for oss. Da det nærmer seg slutten av dagen er det tid for gruppe-refleksjon. Under gruppe-refleksjonen forteller vi litt om hva hver enkelt har jobbet med i løpet av dagen, men samtlige av oss kommer med den samme observasjonen. Det har vært veldig stille i dag.\\ 

\textit{"Jeg synes det er godt å få begynne å jobbe. Få jobbet i fred og ro. Det har vært lite diskusjoner, men for lite informasjonsflyt. Rart å jobbe i en slik setting igjen, tror det er viktig med dialog fremover slik at vi blir mer involvert i arbeidet til hverandre."}{--Håkon}\\

\textit{"Synes i grunn det var litt godt å møte opp for så å bare starte. Det har vært veldig effektivt og jeg føler jeg har fått utrettet mye mer i dag enn tidligere. Likevel har det kanskje ikke vært så mye involvering fra hverandre, og jeg er litt usikker på hva de andre har gjort så langt. Bør vi ha steg i løpet av dagen der man deler hva man har gjort?. Dette tror jeg vil føre til mer involvering, uten at det går for mye ut over effektiviteten."}{--Jonas}\\

\textit{"Jeg synes også det har vært en effektiv dag, men ikke for lite involvering. Dere har fortalt hva dere holder på med, og kommet med små kommentarer fra deres oppgaver underveis. Dette har gitt meg et ganske godt inntrykk av hva som skjer på tvers av gruppa i alle fall.}{--Lars}\\
\\
Det er altså blandede følelser om hvordan vi føler arbeidsdagen har gått. På den ene siden har det vært godt å komme i gang og fått gjort noe, mens på den andre siden har det vært lite involvering og lite kommunikasjon på tvers av gruppa. Vi har en omtrentlig situasjonsoversikt men med den manglende kommunikasjonen er det fare for at vi går glipp av verdifull input som kan løfte oppgaven betraktelig.\\



\subsection{Refleksjoner}


I ettertid har gruppa reflektert over utslaget denne arbeidsfordelingen har gitt, både i forhold til gruppedynamikk men også i forhold til resultatet. Med det klare skillet mellom de to subgruppene har en god sammenføyning av prosjektene vært vanskelig. Vi har ikke opplevd det som problematisk i forhold til å få et godt ferdig produkt. Subgruppene internt har samarbeidet godt og jobbet målrettet. Derimot har behovet for samarbeid mellom subgruppene vært tilnærmet fraværende da de respektive gruppenes arbeid ikke har hatt direkte innvirkning på den andres.\\ 

En viktig del av dette faget er å erfare hvordan det er å jobbe sammen til tross for ulike av bakgrunner, erfaringer og oppgaver. Ved ikke å jobbe så tett sammen har vi redusert muligheten for å lære mer om hverandres fagfelt, men vi samtidig har vi utnyttet gruppas kompetanse på en mest mulig effektiv måte. Dette for å få fremdrift i prosjektet. Ved å ikke bruke tid på å lære opp hverandre har vi kunnet bruke tiden effektivt under de forutsetninger hvert enkelt gruppemedlem har.\\

Grunnen til at vi valgte denne arbeidsfordeling var at både Anders og Jonas følte de måtte ha investert masse tid i å lære det grunnleggende IT- og elektronikktekniske før de kunne startet med selve prototypearbeidet. Dette hadde kostet masse tid, og når selve prototypen ikke direkte er med i vurderingen av prosjektet i sin helhet, hadde dette vært mye tid investert i noe som ikke nødvendigvis hadde høynet kvaliteten på rapporten. Ali havnet midt mellom og har vært inn og ut av begge subgruppene gjennom semesteret. Han har gjort mye research og lært mye, men utbyttet av arbeidet føler han ikke har vært så stort. Han stiller spørsmål til hvor nyttig denne researchen har vært i forhold til den endelige rapporten.\\

Håkon kom med en interessant refleksjon som gikk ut på at det er sluttproduktet vi har fokusert på, ikke å dra mest mulig lærdom av hverandres fagkomptanse. Dette var gruppa enige i. Det er tross alt sluttproduktet vi skal vurderes etter, ikke direkte hvordan vi har kommet fram til målet. Dermed kan man rettferdiggjøre arbeidsfordelingen ved å hevde at det var den mest effektive måten å nå målet på, men med tanke på læringsutbytte til hver enkelt student så var det kanskje ikke den beste fordelingen for å lære av hverandre. \\

En ting vi følte kunne være en ulempe, var at vi ikke stilte nok kritiske spørsmål til hverandre. De to subgruppene hadde tillit til hverandres arbeid og antok at det de andre holdt på med eller kom fram til var nyttig for prosjektet som helhet. Her antar Håkon at dersom vi hadde jobbet mer med det samme, eller i det minste vært mer inkludert i hverandres arbeid, så kunne vi presset hverandre på akkurat \textbf{hvorfor} vi gjorde som vi gjorde. Dette mente Håkon kunne styrket produktene fra hver enkelt subgruppe. For sluttproduktet tenkte Anders på dette positivt. Som mer eller mindre utenforstående for hva de andre gjorde, ga det samme innblikket i stoffet som for noen som leser det for første gang. Dermed når det kom til rapportskriving, kunne man be vedkommende utdype eller reformulere seg slik at innholdet enklere forståes. Dette førte til flere kritiske spørsmål og en rekke omformuleringer når rapportene skulle gjennomgåes før levering.\\

Ved å ikke ha et eierforhold til innholdet er derimot terskelen høyere for å skjære igjennom med store endringer da en ikke kjenner sammenhengen på samme måte som forfatterene. Det samme gjelder arbeidet med prototypen. Uten å ha kjennskap til hvordan de ulike modulene fungerer er det vanskelig å komme med konstruktive tilbakemeldinger og forslag.\\

Ali mente at det ble lite eierfølelse til prosjektet som helhet, når man jobbet med så forskjellige temaer. Det var rimelig effektivt å dele opp gruppen på denne måten, men han følte at det ble gjort nærmest som to separate prosjekt. Lars sa seg enig i dette, og innrømte at han ikke visste mye om hva som hadde blitt gjort på samfunnsdelen av prosjektet. Dette syns Lars var synd, da mye av poenget med EiT var å jobbe sammen. Lars reflekterte rundt dersom vi skulle startet på nytt i dag, kunne vi gjort det annerledes. Han trodde det kunne blitt mer samarbeid dersom vi delte opp de to oppgavene våre i to bolker. Først hadde vi gjort research på Truck Pooling og ferdigstilt alt på samfunnsdelen sammen. Etter dette kunne vi utviklet prototypen i felleskap. Dette var en interessant idé som blir drøftet litt nærmere som en av aksjonene vi kunne tatt.\\

Det at gruppa har fordelt arbeidsoppgavene på denne måten har ført til mangel på samhørighet i gruppa. Dette er ugunstig da samhørighet har en rekke positive effekter innad i en gruppe, se \cite{Promotegroup}:
\begin{itemize}
    \item [-] Økt konformitet.
    \item [-] Økt gruppeinflytelse over øvrige medlemmer.
    \item [-] Hvert medlem er mer fornøyd/tilfreds med gruppa.
    \item [-] Bedre samarbeid.
\end{itemize}

En ikke-sammensveiset gruppe er i faresonen for å ta dårlige avgjørelser. I kompendiet blir dette kalt \textit{groupthink}. Dette kjennetegnes ved\cite{EffectiveTeams}:
\begin{itemize}
    \item [-] Når grupper med hensikt jobber i isolasjon og ikke deler funn eller konklusjoner med andre utenfor gruppa. Da øker sjansen for å ta dårlige avgjørelser.
    \item [-] Hvis gruppelederen kontrollerer diskusjonen og uttrykker sine meninger helt fra starten av.
    \item [-] Hvis gruppa står ovenfor en viktig eller stressende avgjørelse øker tendensen for å ta en rask avgjørelse i den hensikt å redusere stresset. Dette igjen resulterer ofte i at en dårlig avgjørelse blir tatt.
\end{itemize}

Arbeidsfordelingen var ikke en dårlig avgjørelse, men sett i lys av hvordan gruppedynamikken fungerte i ettertid kan det hende gruppa har mistet noen av de positive sidene ved gruppesamarbeid. Vi ser at subgruppene kan kjennetegnes ved det første kriteriet for \textit{groupthink}. Dette blir selvsagt hypotetisk og spekulativt, men kanskje verdt å ta til ettertanke.
\newpage

\subsection{Aksjoner}
For å bedre kommunikasjon, samspill, gruppedynamikk og eierskap til prosjektet innførte gruppa daglige briefinger. Briefingene ble gjort etter innsjekk, før lunsj, en gang mellom lunsj og refleksjoner samt på slutten av dagen før refleksjonene. Alt i alt, fire daglige briefinger. Den første briefingen besto som regel av å informere om hva man tenkte å bruke dagen på, hvilke tema, hvilke deloppgaver osv. Det var lagt opp til at man her skulle få input fra de andre gruppemedlemene og diskutere hva som var viktig å få gjort samt å avgjøre om hva man skulle legge mest vekt på. De andre briefingene var korte statusoppdateringer hvor resten av gruppa fikk inntrykk av hvordan det hadde gått samt progresjonen på alle oppgavene.\\

På denne måten var det enklere for alle medlemmene å skaffe seg en oversikt over hva som skjedde og konkrete utfordringer i hver oppgave. I tillegg fikk alle en bredere samhørighet og eierskap til prosjektet og gruppa følte at dette var et positivt og nødvendig tiltak både i forhold til prosjektet, men også det sosiale. Disse briefingene tok omtrent 40-50 miunutter av arbeidsdagen men resultatet var en felles forståelse av hvordan gruppa lå an totalt sett. Gruppa følte at dette var et positivt tiltak som resulterte i progresjon.\\

En annen aksjon gruppa gjorde var å benytte SCRUM metodikken. Det innebærer blant annet de tidligere nevnte briefingmøtene. For følge opp SCRUM valgte vi å bruke verktøyet Trello. Dette er et verktøy som viser status på prosjektet som beskrevet ovenfor. Ved å bruke dette verktøyet aktivt og holde det oppdatert kunne gruppemedlemmene alltid holde seg oppdatert over status og hva de andre medlemmene holdt på med. Dette gav likevel ikke et komplett, helhetlig bilde, men hjalp gruppemedlemmene i å være oppdaterte på hva den andre subggruppen brukte tiden på.\\





%Consensus side 62/63, ikke 100\% av gruppa trenger å være enige, men 70-80\%. 
%Effektiv metode for problemløsning: Shaw: 
%1.	erkjenne at det er et problem
%2.	diagnosisere problemet
%3.	ta en avgjørelse
%4.	akseptere og implementere avgjørelsen
%alternativt:
%1.	orienteringsfase
%2.	diskusjonsfase
%3.	avgjørelsesfase
%4.	implementeringsfase
%«hvis medlemmene ikke føler de kan bidra fritt med ideene, det vil bli vanskligere for gruppa å lykkes.» s64
%S65: diversity norms: medlemmer er forventet å komme en time før. Normer som dette kan hindre individuell frihet og forårsake «resentment»/avsky. 
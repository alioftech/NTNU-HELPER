\chapter{Grupperfaringer}


\section{Gruppeerfaring 1 - Adskilt arbeid}

Etter gruppa hadde valgt arbeidsoppgave, ble det avgjort at vi skulle gjøre litt forskjellige arbeidsoppgaver grunnet vår forskjellige bakgrunn. Det ble fort enighet om at de med elektro-bakgrunn skulle jobbe med prototypen og de fra maskin tok på seg jobben med det samfunnsmessige aspektet. Da prosjektet gikk mot slutten innså vi at ved å arbeid separert på denne måten, hadde vi lite peiling og kontroll over hva de andre holdt på med. Dette gjorde også at det ikke ble like mye gruppearbeid, men mer individuelt arbeid innad i en gruppe.

\section{Gruppeerfaring 2 - Arbeidsmetodikk}

Da vi startet å arbeide med selve prosjektarbeidet holdt vi på med flere forskjellige deloppgaver samtidig. Siden vi ikke hadde en overordnet leder, da dette gikk på rundgang, var det viktig for hvert enkelt medlem å ha en oversikt over hva gruppa holdt på med. Det var viktig å ha en overordnet forståelse av hvordan fremgangen til gruppa var. Derfor brukte vi Trello. Det er et verktøy som egner seg godt til å gjennomføre arbeidsmetodikken SCRUM. På denne måten hadde vi en oversikt over status og fremtidige oppgaver. Dette fungerte godt i starten av prosjektet, men etter noen uker ble ikke oppgaver oppdatert og metodikken ikke fulgt godt nok. Dette førte til at gruppas oversikt over fremgang og arbeid ikke ble så god som ønskelig. Erfaringen vår er at metodikken er veldig god i teorien, men krever innsats for å fungere. Det gjør at den i praksis er vanskelig å bruke. Vi erfarte hvor vanskelig oversikt i team-arbeid kan være.

\section{Gruppeerfaring 3 - Ingen avbrytelser}

Som de personlige refleksjonene bærer preg av har vi innad i gruppa vært fokusert på å gi plass til hverandre og være obs på å ikke avbryte hverandre. Vi tar fram dette som en av hovederfaringene fra dette semesteret. Det har vært ett nytt fokusområde for alle medlemmene, og det har til tider vært krevende å holde tanker inne og vente til man ikke avbryter medlemmet som snakker. Men erfaringene fra medlemmene er positive til denne regelen/loven som vi ble enige om i samarbeidskontrakten den første landsbydagen. 
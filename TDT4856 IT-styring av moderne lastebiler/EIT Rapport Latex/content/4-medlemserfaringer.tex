\chapter{Medlemserfaringer}


\section{Lars sine 3 hovederfaringer}

\subsubsection*{Erfaring 1 - Sosiale aktiviteter tidlig i prosjekter er viktig}
I starten av arbeidet med Eksperter i Team var det mye fokus på å bli kjent gjennom ulike sosiale leker og øvelser.
For meg som ikke var vant til slik, slo det meg som litt underlig at vi skulle bruke såpass mye tid på aktiviteter som ikke virket innlysende å ha innvirkning på selve prosjektet der og da.
I retrospekt ser jeg verdien av dette. Det er mye lettere å starte arbeidet med prosjektet når alle er litt kjent og du har en idé om hva andre tenker og finner interessant.
Det er mulig jeg ikke har tenkt over dette før, siden jeg i alle tidligere prosjektarbeid har kjent de jeg har arbeidet med. Til senere prosjekter i arbeidslivet ser jeg verdien av dette, da det er sjeldent du kjenner alle du vil jobbe i et prosjekt med. 

\subsubsection*{Erfaring 2 - Gjensidig respekt for tanker og idéer er viktig}
Vi fant tidlig ut at noen av gruppemedlemmene snakker mer og skyter inn flere kommentarer enn andre, inkludert meg selv. På bakgrunn av dette ble vi enige om å la folk snakke ut og få fullføre resonnementene sine uten å bli avbrutt. Gjennom å fokusere på dette gjennom prosjektperioden føler jeg diskusjonene har hatt god kvalitet og at alle føler de har blitt hørt.
De aller fleste argumenter for og imot ulike avgjørelser har blitt presentert på en god og sammenhengende måte, siden man har fått tid og plattform til dette.
Ved å fokusere på dette har jeg blitt mer bevist på helheten av andres argumentasjon.
Jeg tror dette førte til at flere riktige beslutninger ble tatt og at prosjektarbeidet ble lettere.

\subsubsection*{Erfaring 3 - Refleksjoner gir prosjektet bedre fremdrift}
Hele idéen med refleksjonsarbeid var nytt for meg og noe jeg anså som unødvendig i starten. 
Etter at gruppa ble mer sammensveiset ble også refleksjonsarbeidet bedre. Gjennom å ta opp ulike situasjoner gjennom dagen fikk vi diskutert og lagt frem alle sine syn. 
Dette førte igjen til at vi tok flere avgjørelser som bedret prosjektarbeidet og fremdriften i prosjektet.
Gjennom refleksjonene ble vi blant annet enig om å ha korte oppdateringsmøter gjennom dagen. Dette sikret at alle arbeider med noe som bidrar til sluttproduktet. Alle gruppas medlemmer fikk også et større eierskap til hele prosjektet, ikke bare til den delen de arbeidet med.

\section{Anders sine 3 hovederfaringer}

\subsubsection*{Erfaring 1 - Refleksjon kan gi andre forutsetninger}

I motsetning til andre prosjekter hvor det har vært resultatet som har stått i fokus, har det i dette tilfellet vært veien til resultatet som vi skal lære av og fokusere på. Det har vært en annerledes tilnærming til å gjøre et prosjekt, og effektiviteten i prosjektet har ikke vært like høyt for hva som er vanlig for meg, men læringsutbytte har vært større. Selv om læringsutbytte har vært like teknisk som det vanligvis bruker, har læringen vært om meg selv. Hvordan \textit{jeg} fungerer i en gruppe og om hvordan forskjellige faktorer påvirker gruppedynamikken og hvordan gruppedynamikken påvirker arbeidet.

\subsubsection*{Erfaring 2 - Kartlegging kan være produktivt}

I starten av faget brukte vi lang tid på å bli kjent og bestemme oss for hva vi skulle velge som tema. Av erfaring er dette noe som tar lang tid og det er ofte usikkert om man har valgt et godt tema før man faktisk har startet på arbeidet. Vi brukte mye tid i starten på å kartlegge hva vi ønsket å gjøre og hvilke temaer som vi skulle fokusere på. Det at vi brukte tid på dette i starten tror jeg sparte oss for mye tid senere i prosjektet. Etter å ha gjort noe researh kunne gruppa finne hvilken retning vi skulle vinkle arbeidet. Dermed ble det enklere å utelukke punkter som ikke fulgte hovedprofilen, og prioriteringen av hva som var viktig ble enklere. 

\subsubsection*{Erfaring 3 - En munn, to ører}

Fra starten av fastslo vi at vi skulle la hverandre prate ferdig og ikke avbryte hverandre når vi diskuterte. Jeg har personlig tatt dette veldig seriøst og jeg tror ikke jeg har avbrutt noen en eneste gang. Dette har både en positiv og en negativ effekt. Man får presentere det man skulle si i sin helhet, uten påvirkning fra de andre, noe som virket som en fordel for de fleste. Men i løpet av prosjektet har jeg merket at jeg ikke er så god til å ordlegge meg, og at jeg kan ha godt av å få innputt til det jeg snakker om. Likevel mener jeg at man kan komme lengre som gruppe ved å ha tålmodigheten til å høre på andre framfor å avbryte å komme til en beslutning. Det er en grunn til at vi har to ører, men bare en munn.

\section{Ali sine 3 hovederfaringer}

\subsubsection*{Erfaring 1 –  Hvordan gi og få tilbakemeldinger} 
 Dette faget har tvunget oss til å tenke mer på kommunikasjon. Hvordan man kan gi tilbakemelding på riktig og effektiv måte? Dette for at det skal bli tatt imot på en saklig måte, heller enn personlig. Dette er noe jeg føler jeg har fått god innføringen i gjennom semesteret og virkelig kan nå se nytten av. Sent i semesteret hadde vi en øvelse som het 2+1, en øvelse  jeg likte veldig godt. Jeg synes  det var en smart måte å gi tilbakemeldinger på. 
 
\subsubsection*{Erfaring 2 – Det sosiale betyr mye for kvaliteten av arbeid}
Noe jeg skal ta med meg videre og har fått en større forståelse for er betydningen av å ha tid til å bli kjent med hverandres bakgrunn og hvem de er som en person. Det er mange ting jeg har visst jeg ikke var god i, og kommunikasjon mellom de andre har hjulpet meg å kartlegge det og bli mer kjent med meg selv og hvordan jeg oppfører meg i en gruppe. Dette er noe jeg hadde ikke forventet når jeg meldte meg for faget. Man slutter aldri å lære om seg selv eller andre. 
 
 
\subsubsection*{Erfaring 3 - Kommunikasjon og plattform for å organisere arbeid}
I den tidlige fasen av prosjektet ble det foreslått av Håkon å bruke en arbeidsmetodikk som heter SCRUM. Jeg synes denne arbeidsmetodikken har et stort potensiale hvis den er implementert og tatt i bruk på riktig måte. Arbeidsmetodikken kan gi prosjektets fremdrift høy kvalitet, og kommunikasjon og oversikt står sentralt. Når vi møtte opp på starten av landsbydagene diskuterte vi hva som har blitt gjort siden sist, hva vi vil jobbe med videre, og om er det noen hindringer i vegen. Det å ha all informasjon om arbeidsoppgaver tilgjengelig på et sted og å tildele ansvar til enkeltmedlemmer minker sjansen for misforståelse og gir en fin arbeidsflyt utover prosjektet. 

\section{Håkon sine 3 hovederfaringer}

\subsubsection*{Erfaring 1 - Prosjektarbeid uten eierfølelse}
I dette prosjektet har gruppa jobbet veldig adskilt hva angår prototypearbeidet mot rapportskring og utforsking av temaet Truck Platooning. Jeg erfarte at før vi startet med kortere møter hvor vi kartlegger hva vi holder på med og skal gjøre utover dagen, hadde jeg ikke noe forhold til hva de rundt meg arbeidet med. Det ga en følelse av særs lite eierforhold til de ordene og tankene som ble skrevet ned i rapporten og gjorde at motivasjonen min for å opprettholde gruppesamholdet minket. Etter vi startet med kortere møter to ganger daglig for å kartlegge arbeidet vi holder på med og fortsetter med utover dagen stiger motivasjonen betraktelig. Det er godt å høre og se at de andre også jobber og holder en fin fremdrift. Derimot eierfølelsen slet jeg fortsatt med å finne da jeg aldri helt tok meg tid til å lese gjennom og prosessere og bearbeide tankene mine rundt det arbeidet de andre hadde gjort. Jeg tar med meg videre å forsøke å investere mer tid i andres arbeid for å kunne opparbeide meg et større eierforhold til prosjektet og produktet, og holde å meg oppdatert på de delene av prosjektet jeg ikke arbeider med.  

\subsubsection*{Erfaring 2 - Folkeskikk, oppmøte til avtalt tid}
Jeg har erfart at lengden av min egen lunte er kortere enn andres når det kommer til hva jeg ser på som folkeskikk, mot hva andre ser på som ikke fult så farlig. Jeg tenker på å møte opp til avtalt tid. I gruppa har det variert mye hvor gode vi har vært til å møte opp til avtalt tid. Jeg står fast på at det er folkeskikk å møte opp til avtalt tid, hvor andre har åpentbart hatt et annet syn på det. Derfor har jeg erfart hvordan jeg selv håndterer og forholder meg til personer som gjentatte ganger kommer for sent, og hvordan jeg må forholde meg til andres synspunkter. 

\subsubsection*{Erfaring 3 - Å ikke bryte inn}
Jeg liker måten vi har blitt enige om å diskutere på. Det gir meg en veldig ro når jeg får lytte til resonnementer som blir fullført, og jeg får selv si hva jeg ønsker å uten å bli avbrutt. Det fører til at det blir mindre opphete diskusjoner i gruppa. Jeg tror det gir rom for å forstå hverandre bedre når man lar folk bli ferdige, enn å skulle la de mest rappkjeftete bryte inn og ta over og styre diskusjonene i sine bestemte retninger.  
\section{Jonas sine 3 hovederfaringer}

\subsubsection*{Erfaring 1 - Åpenhet og ærlighet varer lengst}

I tidligere prosjekter har jeg latt uenigheter forbli innvendig, siden jeg har tenkt at det ikke er lenge igjen av prosjektet og la oss bare ha det fint. En ting jeg har lært gjennom EiT er at det ofte er bedre å lufte det man har på hjertet så kan man ha en diskusjon rundt det og evt. komme til en løsning. Dersom man likevel ikke har kommet til en løsning, merker jeg man føler seg bedre av å bare si hva man føler. Alle trenger ikke nødvendigvis å være enig, bare man føler seg forstått.

\subsubsection*{Erfaring 2 - Folk er forskjellige}

Etter den lille krangelen mellom Håkon og Ali så jeg to ganske forskjellige reaksjoner til en situasjon, som jeg følte jeg ville reagert på en helt annen måte igjen. Dette ga meg innsyn i at folk virkelig er satt sammen forskjellig, og at det ofte kan være lurt å ha i bakhodet når man snakker eller samarbeider med andre.

\subsubsection*{Erfaring 3 - Å ikke bryte inn}

Tidlig i EiT, når vi skapte samarbeidsavtalen, ble vi enig at når noen snakket så skulle de få snakke ferdig uten at noen skulle bryte inn. Med dette løftet virker det som alle parter \textit{hører} mer etter, og analyserer og tenker gjennom hva som blir lagt fram før man hopper på med et svar. For min del ga det også rom for å tenke \textit{hvorfor} jeg ville svare som jeg svarte, og dermed få en liten ekstra dimensjon i å bedømme om det jeg kom med var bra eller ei. 

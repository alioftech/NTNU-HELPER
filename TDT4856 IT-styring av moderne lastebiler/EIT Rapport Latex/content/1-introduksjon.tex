\chapter{Introduksjon}


Denne rapporten tar for seg gruppemedlemmenes erfaringer, forkunnskaper og forventinger til faget vil bli presentert, samt tre forskjellige samarbeids-caser som har oppstått under arbeid sammen dette semesteret. Det skal diskuteres og reflekteres over disse casene slik at læringsutbytte blir størst mulig. Rapporten konkluderer med tre erfaringer hvert medlem tar med seg videre, samt tre felles gruppeerfaringer.\\

Vi ønsker å takke faglærer Svein-Olaf "Sophus" Hvasshovd for faglig støtte og input gjennom semesteret, Terje Moen ved Sintef Teknologi og Samfunn og Arvid Aakre ved institutt for Bygg og Miljøteknikk for gode samtaler og innspill, samt læringsassistentene Martine Gran og Julian Vollen for veiledning og fasilitering. Vi ønsker også å takke Nordic Semiconductor som bidro med gratis utviklingskort, Omega Verksted for lån av fasiliter og Hackerspace NTNU for lån av byggesett.



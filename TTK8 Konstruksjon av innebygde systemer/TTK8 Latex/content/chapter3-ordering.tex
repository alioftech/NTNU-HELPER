%===================================== CHAP 3 =================================

\chapter{Preparing the PCB for manufacturing}
This chapter explains how to export the gerber files, which are needed by the manufacturer to produce the PCB. As well as creating a BOM (bill of material) to order all of the needed components.


\section{Exporting Gerber files} 
To export the gerber files the Make function in Eagle is used to create a .zip file with the gerbers. 
The files can be found under \textbf{[Installation-folder]/Gerbers} directory and selected layers can be seen in Appendix \ref{board_outlines}.


\begin{table}[h]
\centering
\begin{tabular}{|c|c|}
\hline 
\rowcolor{Gray}
File name & Description \\ 
\hline \\
brd.boardoutline.ger & Helps the manfacturer (Elprint) automatically detect the  \\ 
 & board outline/dimensions \\ 

\hline 
brd.bottomlayer.ger & The copper bottom layer tells the manufacturer where to\\ 
 &  lays copper on the bottom. \\ 
\hline 
brd.bottomsoldermask.ger & Bottom solder mask is a thin layer that protects against oxidations \\ 

 & and helps preventing solder bridges between solder pads. \\ 
\hline 
brd.drills.xln & The drill file contains drilling details for production of the PCB.  \\ 
\hline 
brd.toplayer.ger & Contains the top copper layer. \\ 
\hline 
brd.topsilkscreen.ger & contains text and outline for the components drawn   \\ 
\hline 
brd.topsoldermask.ger & This covers all vias and component traces to protect from corrosion  \\
				     & and accidental electrical shorts\\
\hline
\end{tabular} 
\caption{Gerber files and description}
\label{tb:gerber}
\end{table}
\section{Bil of Materials section}
A bill of material (BOM) can be found in Appendix \ref{tb:BOM}

